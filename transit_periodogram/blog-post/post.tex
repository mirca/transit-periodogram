\documentclass{article}
\usepackage{amsmath, amssymb, bm}
\title{Sweet \texttt{Cython} code for planet search finding}
\author{@\texttt{dfm} \& @\texttt{mirca}}

\begin{document}
\maketitle
\section{Why period search algorithms are important?}
\subsection{Statistical assumptions}
We assume that $y_i \sim \mathcal{N}(m(t_i), \sigma_i)$, $i=1, 2, ..., n$, where
\begin{equation}
m(t_i) =
\left\{
    \begin{array}{ll}
        l, & \text{if}~t_0 \leq t_i \leq t_0 + w ,\\
        h, & \text{otherwise}.
    \end{array}
\right.
\end{equation}
such that $h>l$, in which $t_0$ is the transit time, $w$ is the transit duration,
$h$ is the mean (not quite) flux level out-of-transit, $l$ is the mean (not quite)
flux level in-transit.

Then the log-likelihood function of $\bm{y} \triangleq (y_1, y_2, ..., y_n)$,
assuming $y_i$'s are pair-wise idependent, can be written as
\begin{align}
    \log p(\bm{y}) = -\dfrac{1}{2}\sum_{i=1}^{n}\left(\dfrac{y_i - m(t_i)}{\sigma_i}\right)^2,
\end{align}
up to an additive constant.

Note further that, the log-likelihood function can be expressed as
\begin{align}
    \log p(\bm{y}) = -\dfrac{1}{2}\left[\sum_{i \in I}\left(\dfrac{y_i - l}{\sigma_i}\right)^2
                                        + \sum_{i \in I^{c}}\left(\dfrac{y_i - h}{\sigma_i}\right)^2\right],
\end{align}
in which $I\triangleq\left\{i | t_0 \leq t_i \leq t_0 + w \right\}$ and $I^{c}$ denotes the complement of $I$.

And the maximum likelihood estimator for $h$ and $l$, denoted as $h^{\star}$ and $l^{\star}$,
respectively, can be written as
\begin{equation}
    l^{\star}(\bm{y}) = \dfrac{\displaystyle\sum_{i \in I}\dfrac{y_i}{\sigma^2_i}}{\displaystyle\sum_{i\in I}\dfrac{1}{\sigma^2_i}},
    ~~h^{\star}(\bm{y}) = \dfrac{\displaystyle\sum_{i \in I^{c}}\dfrac{y_i}{\sigma^2_i}}{\displaystyle\sum_{i\in I^{c}}\dfrac{1}{\sigma^2_i}}.
\end{equation}

Note that
\begin{align}
    \mathrm{var}\left(l^{\star}(\bm{y})\right) = \dfrac{\displaystyle\sum_{i \in I}\dfrac{\mathrm{var}(y_i)}{\sigma^4_i}}
    {\displaystyle\left(\sum_{i\in I}\dfrac{1}{\sigma^2_i}\right)^{2}}
    = \displaystyle\left(\sum_{i\in I}\dfrac{1}{\sigma^2_i}\right)^{-1}\\
    \mathrm{var}\left(h^{\star}(\bm{y})\right) = \dfrac{\displaystyle\sum_{i \in I^c}\dfrac{\mathrm{var}(y_i)}{\sigma^4_i}}
    {\displaystyle\left(\sum_{i\in I^c}\dfrac{1}{\sigma^2_i}\right)^{2}}
    = \displaystyle\left(\sum_{i\in I^c}\dfrac{1}{\sigma^2_i}\right)^{-1}
\end{align}

The maximum likelihood for the transit depth, denoted as $d^{\star}$, can then be written as
\begin{equation}
    d^{\star}(\bm{y}) = h^{\star}(\bm{y}) - l^{\star}(\bm{y}).
\end{equation}

Then the variance on $d^{\star}$ can be expressed as
\begin{align}
    \mathrm{var}\left(d^{\star}\right) &= \mathrm{var}\left(h^{\star}\right) + \mathrm{var}\left(l^{\star}\right)\\
    &= \displaystyle\left(\sum_{i\in I}\dfrac{1}{\sigma^2_i}\right)^{-1}
    + \displaystyle\left(\sum_{i\in I^c}\dfrac{1}{\sigma^2_i}\right)^{-1}.
\end{align}

Then the signal to noise ratio of the transit depth can be computed as
\begin{equation}
    \mathrm{SNR}_{d^{\star}}= \dfrac{d^{\star}}{\sqrt{\displaystyle\left(\sum_{i\in I}\dfrac{1}{\sigma^2_i}\right)^{-1}
                                            + \displaystyle\left(\sum_{i\in I^c}\dfrac{1}{\sigma^2_i}\right)^{-1}}}.
\end{equation}

\end{document}

